\chapter{使用Kiali进行流量路由追踪}
在本实验本实验旨在使用Kiali工具对流量路由进行追踪和监控,将使用Kiali进行流量路由追踪,以探索如何在Kubernetes环境中管理流量路由和版本控制。本实验将使用Istio作为服务网格,部署一个具有两个不同版本的前端服务,并使用Kiali来观察流量路由的变化。
\section{实验步骤}
\subsection{创建新的前端部署}
在开始实验之前,我们首先删除已经存在的前端部署。
\begin{lstlisting}[language=bash]
	kubectl delete deploy frontend
\end{lstlisting}
使用YAML文件创建一个新的前端部署,重新创建一个前端deploy,名字还是frontend,但是指定了一个版本标签设置为 original,修改后的yaml文件请见提交的代码。
创建deploy。
\begin{lstlisting}[language=bash]
[root@k8scloude1 ~]# kubectl apply -f frontend-original.yaml 
deployment.apps/frontend created
#deploy创建成功
[root@k8scloude1 ~]# kubectl get deploy | grep frontend
frontend                1/1     1            1           43s
#pod也正常运行
[root@k8scloude1 ~]# kubectl get pod | grep frontend
frontend-ff47c5568-qnzpt                 2/2     Running   0          105s
\end{lstlisting}
创建一个DestinationRule,用于定义两个前端服务的版本:原始版本(original)和新版本(v1)。创建frontend-dr.yaml ,在其中指定subsets,分别为orignal和v1,具体代码请见提交的代码文件。我们使用以下命令创建DestinationRule。
\begin{lstlisting}[language=bash]
	$ kubectl apply -f frontend-dr.yaml
\end{lstlisting}

\subsection{更新VirtualService}

接下来,使用以下命令更新VirtualService,并将所有流量路由到原始版本orginal的前端:
\begin{lstlisting}[language=bash]
kubectl apply -f frontend-vs.yaml
\end{lstlisting}

现在,我们准备创建新版本的前端部署。我们使用以下命令创建名为"frontend-v1"的部署:

\begin{lstlisting}[language=bash]
	$ kubectl apply -f frontend-v1.yaml
\end{lstlisting}

\subsection{更新流量路由权重}

最后,我们将更新VirtualService中的权重,将30%的流量路由到新版本的前端。我们使用以下命令更新VirtualService:

\begin{lstlisting}[language=bash]
	$ kubectl apply -f frontend-30.yaml
\end{lstlisting}
\section{实验结果}
我们通过浏览器访问INGRESS\_HOST,可以观察到前端界面的变化。在刷新页面的过程中,我们可以注意到来自新版本前端(v1)的更新标头和原始版本前端的标头之间的变化。

同时,我们可以使用Kiali工具查看服务的拓扑结构和流量路由。在Kiali界面中选择相应的命名空间和服务,我们可以观察到两个前端服务版本的运行状态以及流量的分布情况。

本实验通过使用Kiali工具对流量路由进行追踪和监控,展示了如何在服务网格中管理多个版本的服务,并灵活地进行流量分配和控制。通过使用Istio的DestinationRule和VirtualService功能,我们可以实现流量路由的精细控制,以满足不同的业务需求。
