\chapter{概述}

\section{选题背景与实验目的}
随着云计算和容器化技术的快速发展,云原生微服务架构已经成为构建现代应用程序的首选方法。微服务架构通过将应用程序拆分为小型、自治的服务,使得团队可以独立开发、部署和扩展各个服务,从而提高开发效率和应用的弹性和可伸缩性。

在微服务架构中,Kubernetes作为一个成熟的容器编排和管理平台,提供了强大的功能和工具,用于自动化应用程序的部署、弹性伸缩、负载均衡和故障恢复等方面。Kubernetes提供了一个统一的平台,简化了应用程序在分布式环境中的部署和管理,使得开发者可以专注于业务逻辑的开发,而不必过多关注底层基础设施的细节。

Online Boutique是一个由Google开发的示例应用程序,旨在演示云原生微服务架构的最佳实践。该应用程序由一组独立运行的微服务组成,模拟了一个电子商务平台,涵盖了用户界面、产品目录、购物车、订单处理等核心功能。通过研究和部署Online Boutique应用,我们可以深入了解微服务架构的实际应用场景,理解微服务之间的通信和协作方式,以及如何在Kubernetes环境中进行部署和管理。

然而,微服务架构也面临着一些挑战,例如熔断、限流、监控、认证、授权、安全和负载等方面的问题。为了解决这些挑战,引入服务网格的概念成为一种解决方案。服务网格框架Istio作为一个开源工具提供了流量管理、安全性、可观测性和策略执行等功能,为微服务架构带来更强大的服务治理能力。

因此,通过研究基于云原生微服务架构的Online Boutique应用,并将其部署在Kubernetes环境中,我们可以全面理解微服务架构、Kubernetes和服务网格技术的应用和实践。这将为我们提升云原生架构设计和实施的能力,使我们在开发和部署现代应用程序时具备更高的技术水平和竞争优势。

本实验的主要目的是:

\begin{itemize}
	\item 研究基于云原生微服务架构的Online Boutique应用,并将其部署在Kubernetes环境中。通过实际部署和运行Online Boutique应用,掌握在Kubernetes中部署和管理微服务应用的技术和方法。
	\item 探索微服务架构中存在的挑战,如熔断、限流、监控、认证、授权、安全和负载等问题,并引入服务网格的概念来解决这些挑战。通过引入Istio作为服务网格框架,实现对微服务的流量管理、安全性和可观测性的增强。
	\item 实践服务拓扑发现、全链路追踪、指标遥测、健康检查等服务治理功能,以实现对Online Boutique应用的全面监控和管理。
	\item 进一步探索在Kubernetes中配置和使用Kiali、Jaeger、Prometheus和Grafana等工具,实现服务网格的可视化、链路追踪、日志监控和健康状态监测。
\end{itemize}

通过完成以上实验目标,我将能够全面了解云原生微服务架构、Kubernetes和服务网格的应用和实践。这将为我提供宝贵的经验,使我们能够更好地设计和实施云原生架构,提升应用程序的可靠性、弹性和可伸缩性,并为开发和部署现代应用程序提供技术上的竞争优势。

\section{实验内容}

本实验将围绕着云原生微服务架构和Kubernetes展开,通过研究和实践基于云原生微服务架构的Online Boutique应用,深入探讨在Kubernetes环境中部署和管理微服务应用的最佳实践。实验将了解微服务架构的设计原则、服务拓扑发现、全链路追踪、指标遥测等关键概念,并将引入Istio作为服务网格框架,以进一步提升微服务的弹性、可观测性和安全性。

总体上,本实验的主要内容可以概括为以下几点:

\begin{enumerate}
	\item 安装和配置Kubernetes集群环境,包括安装Kubernetes和部署Dashboard,以可视化管理集群中的Pod、Service和Deployment。
	\item 将基于云原生微服务架构的Online Boutique应用部署在Kubernetes环境中,确保应用能够成功访问并展示首页。
	\item 使用Istio服务网格框架升级Online Boutique应用,以增强微服务的熔断、限流、监控、认证、授权、安全和负载等方面的能力。实现服务拓扑发现、全链路跟踪、指标遥测和健康检查等服务治理功能。
	\item 在Kubernetes中配置Kiali,实现Istio服务网格的可视化,为Online Boutique项目提供服务拓扑图、全链路跟踪、指标遥测、配置校验和健康检查等功能。
	\item 在Kubernetes中配置Jaeger,为Online Boutique项目提供分布式调用链追踪系统,监控和排查基于微服务的分布式系统问题,如分布式上下文传播、分布式事务监控、根因分析、服务依赖关系分析和性能/延迟优化。
	\item 安装和配置Prometheus、Grafana,实现对整个Kubernetes集群和工作负载的运行情况和健康状态的监控。
	\item 将Online Boutique项目中的非Java微服务改写为Spring Boot微服务,并重新部署到Kubernetes中,利用之前配置的组件进行调试、分析和管理。
	\item 总结和整理各项技术平台的底层实现机制,并提供Online Boutique项目的配置和使用步骤,以便更好地理解和应用这些技术。
\end{enumerate}
