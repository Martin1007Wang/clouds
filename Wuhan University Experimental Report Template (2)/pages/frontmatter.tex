% !Mode:: "TeX:UTF-8"

%%% 此部分需要自行填写: 中文摘要及关键词 

%%% 郑重声明部分无需改动

%%%---- 郑重声明 (无需改动)------------------------------------%
\newpage
\thispagestyle{empty}
\vspace*{20pt}
\begin{center}{\ziju{0.8}\pmb{\songti\zihao{2} 郑重声明}}\end{center}
\par\vspace*{30pt}
\renewcommand{\baselinestretch}{2}

{\zihao{4}%

本人呈交的设计报告,是在指导老师的指导下,独立进行实验工作所取得的成果,
所有数据、图片资料真实可靠。 尽我所知,除文中已经注明引用的内容外,
本设计报告不包含他人享有著作权的内容。
对本设计报告做出贡献的其他个人和集体,
均已在文中以明确的方式标明。本设计报告的知识产权归属于培养单位。\\[2cm]

\hspace*{1cm}本人签名: $\underline{\hspace{3.5cm}}$
\hspace{2cm}日期: $\underline{\hspace{3.5cm}}$\hfill\par}
%------------------------------------------------------------------------------
\baselineskip=23pt  % 正文行距为 23 磅
%------------------------------------------------------------------------------





%%======摘要===========================%
\begin{cnabstract}
\thispagestyle{empty}
本实验旨在研究基于云原生微服务架构的 web 商城应用 Online Boutique,并将其部署在 Kubernetes 环境中。通过实验,我深入研究了微服务架构、服务治理和监控方面的相关技术,并应用于 Online Boutique 项目中。

首先,本实验安装了 Kubernetes 集群环境,并通过部署 Dashboard 实现了集群中的可视化管理。我们探讨了不同安装方式,包括二进制安装、kubeadm 安装、minikube 和 kubesphere,并选择适合需求的安装方式。随后,本实验成功将基于微服务架构的 Online Boutique 应用部署在 Kubernetes 环境中,并验证了应用的正常运行。通过部署的过程,我们了解了如何处理微服务的部署、扩展和管理。为了解决 Online Boutique 在熔断、限流、监控、认证、授权、安全和负载等方面的不足,本实验将其升级到服务网格架构,并启用 Istio 来支持微服务。通过 Istio实现了服务拓扑发现、全链路跟踪、指标遥测和健康检查等功能,实现了更强大的服务治理。

在进一步提升可观测性方面,配置了 Kiali 和 Jaeger,为 Online Boutique 项目提供了服务网格的可视化和分布式调用链追踪系统,能够监控和排查基于微服务的分布式系统问题,并进行根因分析和性能优化。安装和配置了 Prometheus 和 Grafana,实现了对整个 Kubernetes 集群和工作负载的运行情况和健康状态的监控。

最后,本实验尝试将 Online Boutique 项目中的一个非 Java 实现的微服务改写成 Spring Boot 的微服务,并重新使用之前配置的组件进行部署、调试、分析和管理。

本实验报告记录了各项实验过程,总结了各项技术平台的底层实现机制,并提供了在 Online Boutique 项目中的配置和使用步骤。通过该实验,我获得了丰富的实践经验,为今后的云原生微服务架构设计和应用提供了宝贵的参考和指导。

\end{cnabstract}
\par
\vspace*{2em}


%%%%--  关键词 -----------------------------------------%%%%%%%%
%%%%-- 注意: 每个关键词之间用“;”分开,最后一个关键词不打标点符号
\cnkeywords{云原生微服务架构;服务网格;Kubernetes}



